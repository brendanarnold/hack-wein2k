\markup{begin:qtl}
\section[QTL]{QTL (calculate densities of states)}
\label{sec:qtl}
\index{qtl}\index{densities of states}

\prog{qtl} creates input for calculation of total density of states,
 spin projected densities of 
states and densities of states projected on an arbibtrary basis on a
given $l$-subshell of any atom (including the relativistic $jj_z$ basis)
using \prog{tetra}.
The caculation is based on the spectral decomposition of a density matrix
on a given atomic site and its transformation to the required basis.
There are three types of input, the ordinary input file described below, the
unitary transformation matrix from ${}_{lms}$-basis to the required one, and
the proper setting of local rotation matrix in the case.struct file, which 
could be different from the setting for scf calculation. In the special
case of $jj_z$ projected densities of states the local $z$-axis must coincide with 
the magnetization direction defined in case.inso. This is not done automaticaly but
a message is written in the output together with the spin coordinate matrix.
The unitary transformation matrices for the most common bases (e.g. ${}_{jj_z}$, ${}_{lms}$, or
$e_g-t_{2g}$) are supplied with the code. For less common cases these must be generated
by hand. The transfomation matrices are read from file case.cf$iatom and these output
is written to case.qtl$iatom, which is used as an input for \prog{tetra}.
 

\contribution{
   J.Kune\v{s} \\
Inst. of Physics, Acad.Science, Prague, Czeck Republic \\
   email: kunes@fzu.cz \\
}

\subsection{Execution}
\label{sec:qtl-exe}

The program \prog{lapwdm} is executed by invoking the command:
\begin{quote}
  \file{x qtl [ -up/dn -p -c -so ]} or\\
  \file{qtl qtl.def }
\end{quote}


\subsection{Dimensioning parameters}
\label{sec:qtl-dim}

The following parameters are used (collected in file \file{param.inc}):
$$  ??????
\dimpar{
LMAX &    highest l of wave function inside sphere (consistent with
          \prog{lapw1}) \\
LABC &    highest l of wave function inside sphere where SO is considered \\
LOMAX &   max l for local orbital basis\\
NRAD &    number of radial mesh points\\
}

\subsection{Input}
\label{sec:qtl-in}

A sample input for \file{qtl} is given below.

{\scriptsize
\begin{verbatim}
------------------ top of file: case.inq --------------------
FULL               (SUMA,SPIN,TOTA)
DOSYM              (NOSYM)
0.0 1.2                   Emin, Emax
0.768                     Fermi energy 
 2                       number of atoms for which density matrix is calculated
 1  2      index of 1st atom, L
 4  3
------------------- bottom of file ------------------------
\end{verbatim}
}

Interpretive comments on this file are as follows:

\begin{description}
\item[line 1:] format{A4}\\
    \tabthree{
  FULL  & & all $2(2l+1)$ components of $l$-subshell calculated\\
  SUMA  & & olny sums defined by stars in case.cf$n file calculated\\
  SPIN & & projections of total DOS on up/dn subspaces for the case of calculation with SOC\\
  TOTA & & shortcut for calculating total DOS only.\\}
\item[line 2:] format{A5}\\
     \tabthree{
 DOSYM & & standard option \\
 NOSYM & & symmetrization schwiched off. Allowed only in special cases.\\}
\item[line 3:] free format \\
  \tabthree{emin, emax & & energy window\\
\item[line 4:] free format \\
   \tabthree{ef & & Fermi energy\\}
\item[line 5:] free format \\
   \tabthree{natom & & number of atom for which projected DOS is calculated\\}
\item[line 6:] free format \\
   iatom, l \\
   \tabthree{iatom & & index of atom for which projected DOS should be calculated \\
             l & & l-value for which projected DOS should be calculated\\
}
\item[line 6 is repeated natom times]\\
\end{description}
\markup{end:qtl}

