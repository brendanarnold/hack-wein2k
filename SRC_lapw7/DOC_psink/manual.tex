\documentclass[10pt,fleqn,a4paper,twosided]{article}
% last changes: 21.07.00 (ba)
%               12.11.00 (ub) updating the manual
\usepackage{epsfig}
\usepackage{amssymb}
\usepackage{graphics}
\usepackage{fancybox}
\usepackage{german}
\usepackage{color}
\usepackage{latexsym}
\textheight21cm
\textwidth15cm
%\renewcommand{\baselinestretch}{1.5}
\oddsidemargin 1cm
\evensidemargin 0cm
\topmargin -.5cm
\parindent0cm

\begin{document}
%\draft
%\narrowtext
%\sloppy


\subsection{LAPW7 (wave functions on grids / plotting)}
\vspace*{.5cm}

This program was contributed by:

\vspace*{.5cm}
\hspace*{1.5cm}
\fbox{
\begin{minipage}{11.cm}
Uwe Birkenheuer \\[-.0ex]
Max-Planck-Institut f\"ur Physik komplexer Systeme \\[-.0ex]
N\"othnitzer Str.~38, D-01187 Dresden, Germany \\[-.0ex]
email: birken@mpipks-dresden.mpg.de \\[-.0ex]
and \\[-.0ex]
Birgit Adolph, \\[-.0ex]
University of Toronto, T.O., Canada \\[0.2ex]
{\small Please make comments or report problems with this program
        to the authors.}
\end{minipage}}
\vspace*{.75cm}

The program {\tt lapw7} generates wave function data on spatial grids
for a given set of $k$-points and electronic bands. 
{\tt lapw7} uses the wave function information stored in {\tt case.vector} 
(or in reduced (filtered) form in {\tt case.vectorf} which can be obtained
from {\tt case.vector} by running the program {\tt filtvec}).
Depending on the options set in the input file {\tt case.in7(c)} one can
generate the real or imaginary part of the wave functions, it's modulus
(absolute value) or argument, or the complex wave function itself. 
For scalar-relativistic calculations both the large and the small 
component of the wave functions can be generated (only one at a time).
The wave functions are generated on a grid which is to be specified in
the input file(s). The grid can either be any arbitrary list of points
(to be specified free-formatted in a separate file {\tt case.grid}) 
or any $n$-dimensional grid ($n=0...3$). The operating mode and 
grid parameters are specified in the input file {\tt case.in7(c)}.
As output {\tt lapw7} writes the specified wave function data for further
processing -- e.g. for plotting the wave functions with some graphical tools
such as gnuplot -- in raw format to {\tt case.psink}. For quick inspection,
a subset of this data is echoed to the standard output file
{\tt case.outputf} (the amount of data can be controlled in the input).
In case, {\tt lapw7} is called many times for one and the same wave function,
program overhead can be reduced, by first storing the atomic augmentation
coefficients $A_{lm}$, $B_{lm}$ (and $C_{lm}$) to a binary file {\tt case.abc}.
For the spin-polarized case two different calculations have to be performed
using either the spin-up or the spin-down wave function data as input.
 
It should be easy to run {\tt lapw7} in parallel mode, and/or to apply it to
wave function data obtained by a spin-orbit interaction calculation.
None of these options have been implemented so far.
Also, {\tt lapw7} has not yet been adapted for {\bf WIEN in a BOX}.


\subsubsection{Execution}

The program {\tt lapw7} is executed by invoking the command:

\vspace*{.25cm}
\hspace*{1.cm}
{\tt lapw7 lapw7.def}\ \ \ or\ \ \ {\tt lapw7c lapw7.def}\ \ \ or\ \ \ 
{\tt x lapw7 [-c] [-up|dn] [-sel]}

\vspace*{.25cm}
With the {\tt -sel} option {\tt lapw7} expects data from the reduced (filtered) 
wave function file {\tt case.vectorf}, otherwise the standard wave function file
{\tt case.vector} is used. The reduced vector file {\tt case.vectorf} is
assumed to resist in the current working directory, while the standard 
vector file {\tt case.vector} (which may become quite large) is looked for in
the {\tt WIEN} scratch directory. For details see {\tt lapw7.def}.

An example of {\tt lapw7.def} is given below.

\vspace*{.25cm}
{\tt \begin{tabular}{llll}
\multicolumn{4}{c}{
- - - - - - - - - - - - - - - - - top of file - - - - - - - - - - - - - - - - -}\\[-.0ex]
2  & ,'gaas.tmp'     & ,'unknown' & ,'formatted'   \\[-.0ex]
5  & ,'gaas.in7c'    & ,'old'     & ,'formatted'   \\[-.0ex]
6  & ,'gaas.output7' & ,'unknown' & ,'formatted'   \\[-.0ex]
7  & ,'gaas.grid'    & ,'unknown' & ,'formatted'   \\[-.0ex]
8  & ,'gaas.struct'  & ,'old'     & ,'formatted'   \\[-.0ex]
10 & ,'gaas.vectorf' & ,'old'     & ,'unformatted' \\[-.0ex]
12 & ,'gaas.abc'     & ,'unknown' & ,'unformatted' \\[-.0ex]
18 & ,'gaas.vsp'     & ,'old'     & ,'formatted'   \\[-.0ex]
21 & ,'gaas.psink'   & ,'unknown' & ,'formatted'   \\[-.0ex]
\multicolumn{4}{c}{
- - - - - - - - - - - - - - - - - end of file - - - - - - - - - - - - - - - - -}\\
\end{tabular}}


\subsubsection{Dimensioning parameters}

The following parameters are listed in file {\tt param.inc\_(r/c)}:

\vspace*{.25cm}
\begin{tabular}{lll}
NATO  & & number of inequivalent atoms per unit cell \\[-.0ex]
NDIF  & & total number of atoms per unit cell \\[-.0ex]
NRAD  & & number of radial mesh points \\[-.0ex]
NSYM  & & order of point group \\[-.0ex]
NGDIM & & overall number of grid points \\[-.0ex]
NMAT  & & total number of basis functions (matrix size) \\[-.0ex]
NRMT  & & number of different muffin-tin radii \\ [-.0ex]
LMAX7 & & maximum L value used for plane wave augmentation \\[-.0ex]
LOMAX & & maximum L value used for local orbitals \\
\end{tabular}
\vspace*{.25cm}

The meaning of {\tt LMAX7} is the same as that of {\tt LMAX2} in 
{\tt lapw2} and that of {\tt LMAX-1} in {\tt lapw1}. Rather than being 
an upper bound it directly defines the number of augmentation functions 
to be used. It many be set different to {\tt LMAX2} in {\tt lapw2} or 
{\tt LMAX-1} in {\tt lapw1}, but it must not exceed the latter one.
Note that, the degree of continuity of the wave functions across the boundary
of the muffin tin sphere is quite sensitive to the choice of the parameter 
{\tt LMAX7}. A value of 8 for {\tt LMAX7} turned out to be a good compromise.


\subsubsection{Input}

A sample input is given below. It shows how to plot a set of wave 
functions on a 2-dim.\ grid.

\vspace*{.25cm}
{\tt \begin{tabular}{lll}
\multicolumn{3}{l}{
- - - - - - - - - - - - - - - - - top of file - - - - - - - - - - - - - - - - -}\\[-.0ex]
2D ORTHO         & & \textmd{\#} mode\ \ O(RTHOGONAL)|N(ON-ORTHOGONAL) \\[-.0ex]
0 0 0 2          & & \textmd{\#} x, y, z, divisor of origin \\[-.0ex]
3 3 0 2          & & \textmd{\#} x, y, z, divisor of x-end \\[-.0ex]
0 0 3 2          & & \textmd{\#} x, y, z, divisor of y-end \\[-.0ex]
141 101 35 25    & & \textmd{\#} grid points and echo increments \\[-.0ex]
NO               & & \textmd{\#} DEP(HASING)|NO (POST-PROCESSING) \\[-.0ex]
RE\ \ ANG\ LARGE & & \textmd{\#} switch\ \ ANG|ATU|AU\ \ LARGE|SMALL \\[-.0ex]
1 0              & \hspace*{1.5cm} & \textmd{\#} k-point, band index \\[-.0ex]
\multicolumn{3}{c}{
- - - - - - - - - - - - - - - - - end of file - - - - - - - - - - - - - - - - -}\\
\end{tabular}}
\vspace*{.5cm}


Interpretive comments on this file are as follows.

\vspace*{.5cm}
\begin{tabular}{llll}
{\bf line 1:}
        & \multicolumn{3}{l}{format(A3,A1)} \\
        & \multicolumn{3}{l}{mode flag} \\
        & mode   & & the type of grid to be used \\
        &        & ANY
        & \parbox[t]{7.7cm}{An arbitrary list of grid points is used.} \\
        &        &0D, 1D, 2D, or 3D 
        & \parbox[t]{7.7cm}{An $n$-dim.\ grid of points is used. 
                            $n = 0, 1, 2, \mbox{or } 3$.} \\
        & flag   & & orthogonality checking flag (for $n$-dim.\ grids only) \\
        &        & N 
        & \parbox[t]{7.7cm}{The axes of the $n$-dim.\ grid are allowed to
          be non-orthogonal.} \\
        &        & O or $\langle\mbox{blank}\rangle$
        & \parbox[t]{7.7cm}{The axes of the $n$-dim.\ grid have to be mutual
          orthogonal.} \\[18pt]
{\bf line 2:}
        & \multicolumn{3}{l}{free format --- (for $n$-dim.\ grids only)} \\
        & ix iy iz idiv &
        & \parbox[t]{7.7cm}{Coordinates of origin of the grid, where x=ix/idv
          etc.\ in units of the {\it conventional} lattice vectors.} \\[18pt]
{\bf line 3:}
        & \multicolumn{3}{l}{free format --- (for $n$-dim.\ grids with $n>0$ 
          only)} \\
        & ix iy iz idiv &
        & \parbox[t]{7.7cm}{Coordinates of the end points of each grid axis.\\
          This input line has to be repeated $n$-times.} \\[18pt]
{\bf line 4:}
        & \multicolumn{3}{l}{free format --- (not for 0-dim.\ grids)} \\
        & np ... npo ... &
        & \parbox[t]{7.7cm}{In case of an $n$-dim.\ grid, first the number of
          grid points along each axis, and then the increments for the output
          echo for each axis. Zero increments means that only the first and
          last point on each axis are taken. In case of an arbitrary list of
          grid points, the total number of grid points and the increment for
          the output echo. Again a zero increments means that only the first
          and last grid point are taken. Hence, for $n$-dim.\ grids, altogether,
          $2*n$ integers must be provided; for arbitrary lists of grid points
          two intergers are expected.} \\[18pt]
%
\end{tabular}\newpage\begin{tabular}{llll}
%
{\bf line 5:}
        & \multicolumn{3}{l}{format(A3)} \\
%%%     & \multicolumn{3}{l}{tool} \\
        & tool & & post-processing of the wave functions \\
        &      & DEP
        & \parbox[t]{7.7cm}{Each wave function is multiplied by a complex phase
          factor to align it (as most as possible) along the real axis (the 
          so-called DEP(hasing) option).} \\
        &          & NO 
        & \parbox[t]{7.7cm}{No post-processing is applied to the 
                            wave functions.} \\[18pt]
{\bf line 6:}
        & \multicolumn{3}{l}{format(A3,1X,A3,1X,A5)} \\
        & \multicolumn{3}{l}{switch iunit whpsi} \\
        & switch & & the type of wave function data to generate \\
        &        & RE
        & \parbox[t]{7.7cm}{The real part of the wave functions is 
                            evaluated.} \\[0.1ex]
        &        & IM
        & \parbox[t]{7.7cm}{The imaginary part of the wave functions is 
                            evaluated.} \\[0.1ex]
        &        & ABS
        & \parbox[t]{7.7cm}{The absolute value of the wave functions is 
                            evaluated.} \\[0.1ex]
        &        & ARG
        & \parbox[t]{7.7cm}{The argument the wave functions in the complex 
                            plane is evaluated.} \\[0.1ex]
        &        & PSI
        & \parbox[t]{7.7cm}{The complex wave functions are evaluated.} \\
        & iunit & & the physical units for wave function output \\
        &       & ANG
        & \parbox[t]{7.7cm}{{\AA} units are used for the wave functions.} \\
        &       & AU or ATU
        & \parbox[t]{7.7cm}{Atomic units are used for the wave functions.} \\
        & whpsi  & & the relativistic component to be evaluated \\
        &        & LARGE
        &\parbox[t]{7.7cm}{The large relativistic component of wave function
                           is evaluated.} \\[0.1ex]
        &        & SMALL
        &\parbox[t]{7.7cm}{The small relativistic component of wave function
                           is evaluated.} \\[18pt]
{\bf line 7:}
        & \multicolumn{3}{l}{free format} \\
        & \multicolumn{3}{l}{iskpt iseig} \\
        & iskpt &
        & \parbox[t]{7.7cm}{The $k$-points for which wave functions 
          are to be evaluated. Even if the wave function information is read 
          from {\tt case.vectorf}, iskpt refers to the index of the 
          $k$-point in the original {\tt case.vector} file! If iskpt 
          is set to zero, all $k$-points in {\tt case.vector(f)} are 
          considered.} \\[0.1ex]
        & iseig &
        & \parbox[t]{7.7cm}{The band index for which wave functions 
          are to be evaluated. Even if the wave function information is read 
          from {\tt case.vectorf}, iseig refers to the band index in the 
          original {\tt case.vector} file! If iseig is set to zero, all 
          bands (for the selected $k$-point(s)) which can found in 
          {\tt case.vector(f)} are considered.} \\[18pt]
%
\end{tabular}\newpage\begin{tabular}{llll}
%
{\bf line 8:}
        & \multicolumn{3}{l}{format(A4) --- this line is optional} \\
%%%     & \multicolumn{3}{l}{handle} \\
        & handle & & augmentation coefficient control flag \\
        &        & SAVE or STOR(E)
        & \parbox[t]{7.7cm}{Augmentation coefficients are stored in 
          {\tt case.abc}). No wave function data is generated in this case.
          This option is only allowed if a {\it single} wave function is
          selected in the previous input line.} \\[0.1ex]
        &        & READ or REPL(OT)
        & \parbox[t]{7.7cm}{Previously stored augmentation coefficients are
          read in (from {\tt case.abc}). This option is only allowed if the
          {\it same} single wave function as the one who's augmentation 
          coefficients are stored in {\tt case.abc} is selected in the
          previous input line.} \\[0.1ex]
        &        & anything else
        & \parbox[t]{7.7cm}{Augmentation coefficients are generated from the 
                            wave function information in {\tt case.vector(f)}.} \\
\end{tabular}
\vspace*{1.0cm}

On default {\tt lapw7} is provided in a memory saving mode in which 
grid points and wave function data share memory. In that mode multiple
state selection (by {\tt iseig=0} and/or {\tt iskpt=0}) is not possible
and will result in the error
message ''{\tt Multiple states not allowed with EQUIVALENCE (PSI,RGRID)}``.
To allow multiple selection for the price of larger memory requirements
just uncomment the line\newline
\hspace*{1cm}{\tt EQUIVALENCE (PSI,RGRID)}\newline
in subroutine {\tt main.frc} in {\tt SRC\_lapw7} before compilation of the
code. The error message is suppressed automatically if you do so.


%%%%%%%%%%%%%%%%%%%%%%%%%%%%%%%%%%%%%%%%%%%%%%%%%%%%%%%%%%%%%%%%%%%%%%%%%%%%%%

\subsection{FILTVEC (wave function filter / reduction of case.vector)}
\vspace*{.5cm}

This program was contributed by:

\vspace*{.5cm}
\hspace*{1.5cm}
\fbox{
\begin{minipage}{11.cm}
Uwe Birkenheuer \\[-.0ex]
Max-Planck-Institut f\"ur Physik komplexer Systeme \\[-.0ex]
N\"othnitzer Str.~38, D-01187 Dresden, Germany \\[-.0ex]
email: birken@mpipks-dresden.mpg.de \\[-.0ex]
and \\[-.0ex]
Birgit Adolph \\[-.0ex]
University of Toronto, T.O., Canada \\[0.2ex]
{\small Please make comments or report problems with this program
        to the authors.}
\end{minipage}}
\vspace*{.75cm}

The program {\tt filtvec} reduces the information stored in 
{\tt case.vector} files by filtering out a user-specified selection of 
wave functions. Either a fixed set of band indices can be selected which is 
used for all selected $k$-points (global selection mode), or the band
indices can be selected individually for each selected $k$-point
(individual selection mode).
The complete wave function and band structure information for the 
selected $k$-points and bands is transferred to {\tt case.vectorf}.
The information on all other wave functions in the original file is 
discarded. The structure of the generated {\tt case.vectorf} file is 
identical to that of the original {\tt case.vector} file. Hence, it should 
be possible to use {\tt case.vectorf} as substitutes for {\tt case.vector} 
anywhere in the {\bf WIEN} program package. (This has only been tested for 
{\tt lapw7}.and {\tt filtvec}.)
To filter vector files from spin-polarized calculations, {\tt filtvec}
has to be run separately for both the spin-up and the spin-down files.

{\tt filtvec} has not yet been adapted for {\bf WIEN in a BOX}.


\subsubsection{Execution}

The program {\tt filtvec} is executed by invoking the command:

\vspace*{.25cm}
\hspace*{1.cm}
{\tt filtvec filtvec.def}\ \ \ or\ \ \ {\tt filtvecc filtvec.def}\ \ \ or\ \ \ 
{\tt x filtvec [-c] [-up|dn]}

\vspace*{.25cm}
In accordance with the file handling for {\tt lapw1} and {\tt lapw7} the input 
vector file {\tt case.vector} is assumed to be located in the {\tt WIEN}
scratch directory, while the reduced output vector file {\tt case.vectorf} is
written to the current working directory. See {\tt filtvec.def} for details.

\vspace*{0.25cm}
An example of {\tt filtvec.def} is given below.

\vspace*{.25cm}
{\tt \begin{tabular}{llll}
\multicolumn{4}{c}{
- - - - - - - - - - - - - - - - - top of file - - - - - - - - - - - - - - - - -}\\[-.0ex]
5  & ,'gaas.infc'    & ,'old'     & ,'formatted'   \\[-.0ex]
6  & ,'gaas.outputf' & ,'unknown' & ,'formatted'   \\[-.0ex]
10 & ,'/scratch/gaas.vector'  & ,'old'     & ,'unformatted' \\[-.0ex]
12 & ,'gaas.vectorf' & ,'unknown' & ,'unformatted' \\[-.0ex]
20 & ,'gaas.struct'  & ,'old'     & ,'formatted'   \\[-.0ex]
\multicolumn{4}{c}{
- - - - - - - - - - - - - - - - - end of file - - - - - - - - - - - - - - - - -}\\
\end{tabular}}
\vspace*{.25cm}


\subsubsection{Dimensioning parameters}

The following parameters are listed in file {\tt param.inc\_(r/c)}:

\vspace*{.25cm}
\begin{tabular}{lll}
NMAT  & & total number of basis functions (matrix size) \\[-.0ex]
NUME  & & number of bands \\[-.0ex]
NKPT  & & number of $k$-points \\ [-.0ex]
LMAX  & & maximum number of L values used (as in {\tt lapw1}) \\[-.0ex]
LOMAX & & maximum L value used for local orbitals (as in {\tt lapw1}) \\
\end{tabular}
\vspace*{.25cm}

The parameter {\tt LMAX} and {\tt LOMAX} must be set precisely as in 
{\tt lapw1}; all other parameters must not be chosen smaller than the
corresponding parameters in {\tt lapw1}.

\subsubsection{Input}

Two examples are given below. The first uses global selection mode; the 
second individual selection mode.\vspace{0.5cm}

{\ \ I. Global Selection Mode}

\vspace*{.25cm}
{\tt \begin{tabular}{llll}
\multicolumn{4}{c}{
- - - - - - - - - - - - - - - - - top of file - - - - - - - - - - - - - - - - -}\\[-.0ex]
3 & 1 17 33  & & \textmd{\#} number of k-points, k-points \\[-.0ex]
2 & 11 -18   & & \textmd{\#} number of bands, band indices \\[-.0ex]
\multicolumn{4}{c}{
- - - - - - - - - - - - - - - - - end of file - - - - - - - - - - - - - - - - -}\\
\end{tabular}}
\vspace*{.5cm}

Interpretive comments on this file are as follows.

\vspace*{.25cm}
\begin{tabular}{lll}
{\bf line 1:}
        & \multicolumn{2}{l}{free format} \\
        & kmax\ \ ik(1) ... ik(kmax)
        & \parbox[t]{9.25cm}{Number of $k$-point list items, followed by 
          the list items themselves. Positive list items mean selection of 
          the $k$-point with the specified index; negative list items mean
          selection of a range of $k$-points with indices running from
          the previous list item to the absolute value of the current one. 
          E.g.~the sequence 2~-5 stands for 2, 3, 4, and 5.} \\[18pt]
{\bf line 2:}
        & \multicolumn{2}{l}{free format} \\
        &  nmax $\,$ ie(1) ... ie(nmax)
        & \parbox[t]{9.25cm}{Number of band index items, followed by 
          the list items themselves. Again, positive list items mean selection
          of a single band index; negative list items mean selection of 
          a range of band indices.} \\
\end{tabular}
\vspace*{.5cm}


{\ \ II. Individual Selection Mode}

\vspace*{.25cm}
{\tt \begin{tabular}{lllll}
\multicolumn{5}{c}{
- - - - - - - - - - - - - - - - - top of file - - - - - - - - - - - - - - - - -}\\[-.0ex]
2 : &      &             & & \textmd{\#} number of k-points \\[-.0ex]
17  & \ 4\ & 11 13 15 17 & & \textmd{\#} k-point, number of bands, band indices \\[-.0ex]
33  & \ 3\ & 11 -14 18   & & \textmd{\#} k-point, number of bands, band indices \\[-.0ex]
\multicolumn{5}{c}{
- - - - - - - - - - - - - - - - - end of file - - - - - - - - - - - - - - - - -}\\
\end{tabular}}
\vspace*{.5cm}

Interpretive comments on this file are as follows.

\vspace*{.25cm}
\begin{tabular}{lll}
{\bf line 1:}
        & \multicolumn{2}{l}{free format} \\
        & kmax 
        & \parbox[t]{8.8cm}{the number of individual $k$-points to be
          selected. This number must be followed by any text, e.g.\ 
          'SELECTIONS' or simply ':', to indicate individual selection
          mode.} \\[18pt]
{\bf line 2:}
        & \multicolumn{2}{l}{free format} \\
        & ik\ \ nmax\ \ ie(1) ... ie(nmax)
        & \parbox[t]{8.8cm}{First the index of the selected $k$-point, 
          then the number of band index items, followed by the list items for 
          the current $k$-point themselves. Positive list items mean 
          selection of the band with the specified index; negative list items 
          mean selection of a range of band indices running from the previous 
          list item to the absolute value of the current one. E.g.~the 
          sequence 3~-7 stands for 3, 4, 5, and 7.\\
          This input line has to be repeated {\it kmax}-times.} \\
\end{tabular}
\vspace*{.5cm}

\end{document}
